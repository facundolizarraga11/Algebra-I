\documentclass{article}
\usepackage{amsmath}
\usepackage{mathtools}
\usepackage{graphicx} % Required for inserting images
\usepackage{amsfonts} % Para \mathbb{R}
\usepackage{amssymb}  % Opcional (símbolos extra)
\usepackage{enumitem} % Para personalizar listas
\usepackage[utf8]{inputenc}
\usepackage{xcolor} % Para colores


\definecolor{intersectioncolor}{RGB}{0,100,0} % Verde oscuro
\newcommand{\highlight}[1]{\textcolor{intersectioncolor}{\textbf{#1}}} % Comando personalizado


\title{Algebra 1 - Practica 1}
\author{Facundo Lizarraga}
\date{August 2025}

\begin{document}

\maketitle

\section{Conjuntos}


\begin{enumerate}
    \item Dado el conjunto $A = \{1, 2, 3\}$, determinar cuáles de las siguientes afirmaciones son verdaderas.
    \begin{enumerate}[label=\alph*)]
        \item $1 \in A$ \textbf{(VERDADERO)}
        \item $\{1\} \subseteq A$ \textbf{(VERDADERO)}
        \item $\{2,1\} \subseteq A$ \textbf{(VERDADERO)}
        \item $\{1,3\} \in A$ \textbf{(FALSO)}
        \item $\{2\} \in A$ \textbf{(FALSO)}
    \end{enumerate}

    \item Dado el conjunto $A = \{1, 2, \{3\}, \{1, 2\}\}$, determinar cuáles de las siguientes afirmaciones son verdaderas.
    \begin{enumerate}[label=\alph*)]
        \item $3 \in A$ \textbf{(FALSO)}
        \item $\{3\} \subseteq A$ \textbf{(FALSO)}
        \item $\{3\} \in A$ \textbf{(VERDADERO)}
        \item $\{\{3\}\} \subseteq A$ \textbf{(VERDADERO)}
        \item $\{1,2\} \in A$ \textbf{(VERDADERO)}
        \item $\{1,2\} \subseteq A$ \textbf{(VERDADERO)}
        \item $\{\{1,2\}\} \subseteq A$ \textbf{(VERDADERO)}
        \item $\{\{1,2,3\}\} \subseteq A$ \textbf{(FALSO)}
        \item $\emptyset \in A$ \textbf{(FALSO)}
        \item $\emptyset \subseteq A$ \textbf{(VERDADERO)}
        \item $A \in A$ \textbf{(FALSO)}
        \item $A \subseteq A$ \textbf{(VERDADERO)}
    \end{enumerate}

    \item Dados $A$ y $B$, decir si $A \subseteq B$.
    \begin{enumerate}[label=\alph*)]
        \item $A=\{1,2,3\}$, $B=\{5,4,3,2,1\}$ \textbf{(VERDADERO)}
        \item $A=\{1,2,3\}$, $B=\{1,2,\{3\},-3\}$ \textbf{(FALSO)}
        \item $A=\{x \in \mathbb{R} \mid 2 < |x| < 3\}$, $B=\{x \in \mathbb{R} \mid x^2<3\}$ \textbf{(FALSO)}
        \item $A=\{\emptyset\}$, $B=\emptyset$ \textbf{(FALSO)}
    \end{enumerate}

    \item Dados los subconjuntos:
    
    \begin{itemize}
        \item $A = \{1,\ -2,\ 7,\ 3\}$
        \item $B = \{1,\ \{3\},\ 10\}$
        \item $C = \{-2,\ \{1, 2, 3\},\ 3\}$
    \end{itemize}
    
    del conjunto referencial:
    
    \[ V = \{1,\ \{3\},\ -2,\ 7,\ 10,\ \{1, 2, 3\},\ 3\}, \]
    
    hallar:
    \begin{enumerate}
        \item $A \cap (B \triangle C)$
        
        Primero buscamos la diferencia simétrica entre $B$ y $C$. En este caso, $B$ y $C$ son conjuntos disjuntos (no comparten elementos comunes), por lo que es la union:
        \[
        (B \triangle C) = \{1,\ \{3\},\ 10,\ -2,\ \{1, 2, 3\},\ 3\}
        \]
        Luego tomamos la interseccion con A y nos queda:
        \[
        A \cap (B \triangle C)=\{1,-2,7,3,10,\{3\},\{1,2,3\}\}
        \]
        
        \item $(A \cap B) \triangle (A \cap C)$
        
        Primero calculamos las intersecciones correspondientes:
        \begin{align*}
            A \cap B &= \{1\} \\
            A \cap C &= \{-2, 3\}
        \end{align*}
        
        Luego realizamos la diferencia simétrica entre los conjuntos obtenidos:
        \[
            (A \cap B) \triangle (A \cap C) = \{1, -2, 3\}
        \]
        
        La diferencia simétrica incluye todos los elementos que están en exactamente uno de los dos conjuntos (no en ambos). En este caso, como no hay elementos comunes entre $\{1\}$ y $\{-2, 3\}$, el resultado es su unión.
    
        
        \item $\overline{A} \cap \overline{B} \cap \overline{C}$
                
        \begin{enumerate}
            \item Conjuntos originales:
            \begin{align*}
                A &= \{1, -2, 7, 3\} \\
                B &= \{1, \{3\}, 10\} \\
                C &= \{-2, \{1, 2, 3\}, 3\}
            \end{align*}
            
            \item Complementos respecto al universal $V = \{1, \{3\},-2, 7, 10, \{1, 2, 3\},3\}$:
            \begin{align*}
                \overline{A} &= \{3\}, 10,\{1, 2, 3\} \\
                \overline{B} &= \{-2, 7, \{1, 2, 3\}, 3\} \\
                \overline{C} &= \{1, 7, 10, \{3\}\}
            \end{align*}
            
            \item Intersección paso a paso:
            \begin{align*}
                \overline{A} \cap \overline{B} &= \{\highlight{\{3\}}, 10, \highlight{\{1, 2, 3\}}\} \cap \{-2, 7, \highlight{\{1, 2, 3\}}, 3\} \\
                &= \{\highlight{\{1, 2, 3\}}\} \quad \text{(Elemento común resaltado)} \\
                \overline{A} \cap \overline{B} \cap \overline{C} &= \{\{1, 2, 3\}\} \cap \{\highlight{1}, 7, 10, \highlight{\{3\}}\} \\
                &= \emptyset \quad \text{(No hay elementos comunes)}
            \end{align*}
        \end{enumerate}
    \end{enumerate}
    
    
    \item Dados subconjuntos A, B, C de un conjunto referencial V , describir $(A \cup B \cup C)^c$ en terminos de intersecciones y complementos, y \\
        $(A \cap B \cap C)^c$ en terminos de uniones y complementos. 
    
        \begin{align*}
        (A \cup B \cup C)^c &= A^c \cap B^c \cap C^c \quad \text{(Ley de De Morgan para uniones)} \\
        (A \cap B \cap C)^c &= A^c \cup B^c \cup C^c \quad \text{(Ley de De Morgan para intersecciones)}
        \end{align*}

    \item 
    \item 
    \item Hallar $\mathcal{P}(A)$
        \begin{enumerate}
            \item $A=\{1\}$
            \[
            \mathcal{P}(A) = \big\{ \emptyset,\, \{1\}\, \big\}
            \]
            \item $A=\{a,b\}$
            \[
            \mathcal{P}(A) = \big\{ \emptyset,\, \{a\},\, \{b\},\, \{a,b\} \big\}
            \]
            \item $A=\{1,\{1,2\},3\}$
            \[
            \mathcal{P}(A) = \Big\{
                \emptyset,\ 
                \{1\},\ 
                \{\{1, 2\}\},\ 
                \{3\},\ 
                \{1, \{1, 2\}\},\ 
                \{1, 3\},\ 
                \{\{1, 2\}, 3\},\ 
                A
            \Big\}
            \]
            
        \end{enumerate}

     \item Sean \( A \) y \( B \) conjuntos. Probar que:
    \[
    \mathcal{P}(A) \subseteq \mathcal{P}(B) \iff A \subseteq B
    \]
    
    \textbf{Demostración} (\textit{por doble implicación}):
    
    \begin{enumerate}[label=\roman*)]
        \item (\(\Rightarrow\)) \textit{Ida}: Supongamos que \( \mathcal{P}(A) \subseteq \mathcal{P}(B) \). 
        
        Como \( A \in \mathcal{P}(A) \) (por definición de conjunto potencia), entonces \( A \in \mathcal{P}(B) \). Esto implica que \( A \subseteq B \) (por definición de \( \mathcal{P}(B) \)).
        
        \item (\(\Leftarrow\)) \textit{Vuelta}: Supongamos ahora que \( A \subseteq B \). 
        
        Sea \( X \in \mathcal{P}(A) \). Por definición, \( X \subseteq A \). Como \( A \subseteq B \), por transitividad \( X \subseteq B \), luego \( X \in \mathcal{P}(B) \). Por tanto, \( \mathcal{P}(A) \subseteq \mathcal{P}(B) \).
    \end{enumerate}
    
    \hfill \(\square\) % Símbolo de fin de demostración


    \item 

\end{enumerate}
    

\end{document}
